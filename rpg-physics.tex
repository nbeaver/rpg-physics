\documentclass[12pt,letterpaper]{article}
\usepackage[utf8]{inputenc}
\usepackage{amsmath}
\usepackage{amsfonts}
\usepackage{amssymb}
\usepackage{siunitx}
\usepackage{units}
\author{Nathaniel Beaver}
\title{Physics problems inspired by role-playing games like Dungeons \& Dragons}
\begin{document}
\maketitle

Initial version from junior year of high school, fall 2006 through early spring 2007.

The difficult level varies between middle school and high school level, with some introductory Newtonian physics (including optimization and full-blown two-dimensional kinematics) toward the end.

\section{Map-drawing headaches: constraints on room dimensions}

You are designing a room in a dungeon for an upcoming campaign. Its total area must be 200 square feet (because of an area trap), and one side must be 10 feet longer than the others because of architectural considerations. Find the length of a side.

\subsection{Solution}

\[x(x+10) = 200\]
\[x^2 + 10x - 200 = 0\]
\[(x+20)(x-10) = 0\]

Thus $x = 10$ feet, and the room should be $10$ feet $\times$ $20$ feet.

The dungeonmaster could also construct a room -20 feet by -10 feet, but constructing rooms in negative space makes maps pretty complicated.

\section{Elongated ooze}

An ooze is chasing a hapless adventurer. The adventurer ducks into a 5 foot by 5 foot crawlspace. If the ooze has a volume of 10 ft by 10 ft by 10 ft = 1,000 $\textrm{ft}^3$, how long will it be once it fully enters the crawlspace?

\subsection{Solution}

\[
\frac{1000 \textrm{ ft}^2}{1}
\cdot
\frac{1}{5 \textrm{ ft}}
\cdot
\frac{1}{5 \textrm{ ft}}
=
40 \textrm{ feet long}
\]

The adventurer is advised to use a line-effect spell.

\section{Taking a grenade}

A self-sacrificing adventurer jumps on an exploding magical bomb.

\begin{enumerate}
\item If the damage at 10 feet is 5 damage and the blast radius is about 2.5 feet, how much damage does the poor guy take, assuming an inverse square law?
\item If the bomb has an energy of 1 ounce of TNT ($\approx 130$ kJ) and half of it is transferred to upward kinetic energy, how high is the adventurer launched, assuming a mass of 100 kg including equipment?
\end{enumerate}

% Note: Original version had an energy of 1 kJ, which is way too low, and also got the inverse square wrong (it used an inverse square with exponential decay), so I've corrected them both.

\subsection{Solution}

%TODO use siunitx to clean this up.

\begin{enumerate}
\item The radius is $10/2.5 = 4$ times smaller, so the damage is $4^2 = 16$ times greater, or $16 \times 5 = 80$. (Note that this assumes the adventurer absorbs all the energy and limits the blast radius to a hemisphere five feet across, which isn't very realistic, but see next part.)
\item $\frac{1}{2} 130$ kJ imparted to 100 kg gives a velocity of $v = \sqrt{2 E/m} = \sqrt{2 65 \textrm{kJ} / 100 \textrm{kg}} \approx 36 \textrm{m/s}$, which will top out at $y = v_0^2 / 2g = (36 \textrm{m/s})^2/ 2 9.8 \textrm{m/s}^2 = 66 \textrm{ meters} = 216 \textrm{ feet}$.
\end{enumerate}

\section{Electric trap}

% The original problem gave a contact time and asked for total charge, but I think this version makes more sense.

Drinort the kobold rogue has just finished swimming across an underground lake to get to some treasure. Unfortunately, in his haste, he neglects to check for traps. If the trap's potential difference is 60 VDC and Drinort's resistance is only 1000 Ohms (since he is wet), how much current flows through his warty body?

\subsection{Solution}

\[I = \frac{V}{R} \]

\[I = \frac{60 V}{1000 \Omega}\]

Thus $I = 60$ mA, which will definitely cause involuntary muscular contractions and could trigger cardiac arrest.

\section{Sandworm escape}

Our hapless adventurer is being chased by a sandworm. He is running towards a small crevice in a nearby cliff, where the worm cannot reach. If the worm can move at 40 ft/s, the adventurer has a 30 ft head start, and the adventurer still needs to run another 100 feet to reach the cliffs, how fast must the adventurer run to reach the cliffs before the sandworm?

\subsection{Solution}
\[t_{worm} = \frac{130 \textrm{ft}}{40 \textrm{ft/s}}\]
\[t_{adventurer} = \frac{100 \textrm{ft}}{v_{adventurer}}\]
\[t_{worm} = t_{adventurer} \implies v_{adventurer} = 40 \textrm{ft/s} \frac{100 \textrm{ft}}{130 \textrm{ft}} \]
Thus $v_{adventurer} \geq$ 30.7 ft/s or roughly 21 miles per hour.

\subsection{Addendum}
Trained Olympic sprinters can hit this speed on 100 meter dashes and the like;
Usain Bolt reached nearly 28 mph and averaged over 23 mph on a 100 meter dash.
The adventurer only has to keep it up for 30.5 meters and is running to save his skin, but it will still be a tough haul.

\section{Griffin rescue}
Galix the griffin is trying to save Elira the elf-maiden who has just been knocked off a cliff. If the cliff is 50 meters deep, Galix is 50 meters above Elira, and Galix can dive at 20 m/s plus the gravitational acceleration, will he save Elira?

\subsection{Solution}

\subsection{Addendum}
So, yes, he will catch her before she hits the bottom, but whether or not he can safely stop himself and her before hitting the bottom is unlikely. Fortunately, Elira had the sense to prepare feather-fall, so she's going to be fine.

\section{Sniper shot}

Keldron the ranger is making a sneak-attack crossbow shot at an orc captain 100 meters away. Keldron's crossbow bolts travel at 40 m/s. 

\begin{enumerate}
\item Assuming Keldron wants to hit the orc at the same height as the bolts leave his crossbow, how high above the target should Keldron aim? How many degrees above the target is this? (Assume a shot at an angle less than 45 degrees.)
\item If the orc captain is moving towards Keldron at 10 m/s on a warg, and Keldron wishes to shoot him through the eye, exactly 0.5 meters higher than the crossbow, now how high above the target should Keldron aim for?
\end{enumerate}

\section{Pillar resonances}

Gandrin the wizard is trying to bring down the temple of Fulsomae to its knees and on top of a certain sorcerer's head.

If the pillars holding up the temple are 10 feet high and made of glass (this was before OSHA and building codes, after all), what frequency should Gandrin set his sonic blast at to excite the first harmonic and shatter the pillars? Treat the pillar as a closed-end air column with $v_{sound}$ = 5170 m/s.

\section{Throw it and run}

Kragen the orc barbarian has a little suprise:
a firebomb hidden in his knickers.

If he throws it up at 10 m/s from his height of 2 meters,
how long does he have till it hits the ground?

If the explosive radius is 10m and Kragen can sprint at 5 m/s, will he get scorched?

What minimum speed must he run to avoid being scorched?

Bonus:
Suppose he doesn't manage to toss it exactly vertically. 
What angle would bring it down on his head,
assuming he throws the firebomb,
then begins running at 5 m/s immediately afterward?

\section{Galilean relativity}

Kragen is riding a warg at 8 m/s,
and again throws a firebomb straight up at 10 m/s,
this time from 3 m since the warg boosts him up.
Due to his imperfect understanding of physics,
he thinks that by throwing the firebomb directly upwards,
it will land on the spot he released it.
In reality, it will keep going at the same velocity as the him and the warg.

\begin{enumerate}
\item If Kragen accelerates, he can be more than 10 feet away from the bomb when it hits the ground. What minimum average acceleration must he maintain? What is his final velocity?
\item Suppose Kragen has learned from experience and now factors in relative velocities. He decides to throw it upwards and backwards. What is the optimal angle (with respect to the vertical) to give him the most time to escape before the bomb hits the ground?
\end{enumerate}


\end{document}